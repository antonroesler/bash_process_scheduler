
\documentclass[runningheads]{llncs}



\title {Visualisierung von Scheduling Algorithmen}
\author{Patrick Frech \and
Leonard Husske\and
Anton Roesler}
\institute {Frankfurt University of Applied Sciences}
\begin{document}
\maketitle

%
\section{Ziel und Ergebnis}
\section{Sruktur}
\section{Implementierung}
\subsection{Grundstruktur zur Speicherung von Prozessinformationen}
Damit der Benutzer einen beliebige Menge von Prozessen in der Simulation erstellen kann, werden die Parameter der Prozesse in dynamischen Arrays gespeichert. Über einen Prozess gibt es wie oben erwähnt folgende Informationen: 

\begin{itemize}
\item Prozessname 
\item Burst time
\item Arrival time
\end{itemize}

Für jeden der genannten Punkte wird ein eigenes Array angelegt. Dabei gehören immer die Daten mit dem selben Index im jeweiligen Array zueinander. 




\begin{table}
\caption{Vom Benutzer gewählte Parameter}\label{tab1}
\begin{tabular}{|l|l|l|}
\hline
Prozessname &  Burst Time & Arrival Time\\
\hline
Pa &  11 & 3\\
Pb & 7 & 13\\
\hline
\end{tabular}
\end{table}




\noindent Folgender Maßen werden die Informationen im Programm gespeichert:

\begin{equation}
process names = [ Pa, Pb ] 
\end{equation}

\begin{equation}
burst time = [ 11, 7 ] 
\end{equation}

\begin{equation}
arrival time = [ 3, 13 ]
\end{equation}



Im weitern Verlauf wird ein Array mit den Process IDs automatisch erstellt. Dieses enthält lediglich die Indexe,
bei \textit{n}-Prozessen, enthält dieses Array die Zahlen von \textit{0} bis \textit{n-1}. Es dient lediglich dazu die o.g. Array einfacher zu iterieren.


%
% the environments 'definition', 'lemma', 'proposition', 'corollary',
% 'remark', and 'example' are defined in the LLNCS documentclass as well.
%

\end{document}
